% neomake: skip
\documentclass{article}

\usepackage{cmap}
\usepackage{array}
\usepackage{enumitem}
\usepackage{makecell}
\usepackage[T2A]{fontenc}
\usepackage[utf8]{inputenc}
\usepackage[english,russian]{babel}
\usepackage{indentfirst}
\usepackage{amssymb}
\usepackage{amsmath}
\usepackage{multicol}
\usepackage{fontawesome}
\usepackage{verbatim}

\usepackage{geometry}
\geometry{top=10mm}
\geometry{bottom=10mm}
\geometry{left=15mm}
\geometry{right=15mm}

\pagenumbering{gobble}

\usepackage[dvipsnames]{xcolor}
\usepackage[colorlinks  = true,
            linkcolor   = blue,
            urlcolor    = blue,
            citecolor   = black,
            anchorcolor = black]{hyperref}
            
\newif\ifen
\newif\ifru

\newcommand{\en}[1]{\ifen#1\fi}
\newcommand{\ru}[1]{\ifru#1\fi}

\rutrue
% \entrue

\renewcommand{\maketitle}{
    \Huge
    \begin{center}
        \textbf{
        \ru{Басалов Ярослав}
        \en{Basalov Iaroslav}
        }
    \end{center}

    \small
    \begin{center}
    % \faMobile \hspace{0.1cm} $\boldsymbol{+}$7(961)260-08-01 
    \faMobile\hspace{0.1cm} \href{tel:89534322053}{$+$7(961)260-08-01}
    \hfill
    \faEnvelope\hspace{0.1cm} \href{mailto:BasalovYA@my.msu.ru}{BasalovYA@my.msu.ru}
    \hfill
    \faPaperPlane\hspace{0.1cm} \href{https://t.me/yabasalov}{@yabasalov}
    \hfill
    \faGithub\hspace{0.1cm} \href{https://github.com/Marilius}{Marilius}
    \end{center}
}

\usepackage{titlesec}
\titleformat{\section}{\Large\bf\raggedright}{}{0.5em}{}[{\titlerule[1pt]}]
\titlespacing{\section}{0pt}{3pt}{7pt}
\titleformat{\subsection}{\large\bfseries\raggedright}{}{0em}{\underline}%[\rule{3cm}{.2pt}]
\titlespacing{\subsection}{0pt}{7pt}{7pt}


\newcommand{\entry}[3]{
	\begin{tabular}{ c | c }
    \begin{minipage}{0.05\linewidth}
    	\begin{center}
    		#1
    	\end{center}
    \end{minipage} 
    &
    \begin{minipage}{0.85\linewidth}
        \textbf{#2} \\ \footnotesize{#3}
    \end{minipage}
    \end{tabular}
}

\newcommand{\interval}[2]{#1 - #2}


\setlist[itemize]{noitemsep, topsep=0pt}

\begin{document}
    \maketitle
    \small

    \section{\ru{Работа}\en{Work}}
        \entry{\interval{2024}{2024}}
        {TechSpace -- \ru{Разработчик}\en{Engineer}}
        {\ru{Разработка на Python}\en{Python development}: 
        	\begin{itemize}
        		\item \ru{Формализация и алгоритмизация поставленных задач для дальнейшего написания программного кода}\en{TODO}.
        		\item \ru{Разработка, реализация, бэктестирование и анализ трейдинговых стратегий на Python3}\en{TODO}.
        	\end{itemize}
        }

        \vspace{.5em}
        
        \entry{\interval{2023}{2023}}
        {\ru{Яндекс}\en{Yandex} -- \ru{Команда Search as a Service}\en{Search as a Service} -- Backend \ru{Разработчик}\en{Engineer}}
        {\ru{Бэкенд разработка на Python}\en{Backend Python development}: 
        	\begin{itemize}
        		\item \ru{Автоматизации и отладка инфраструктурных компонент SaaS}\en{Automation and debugging of SaaS infrastructure components};
                \item \ru{Разработка интерфейса взаимодействия с этими инструментами автоматизации на Flask и aiohttp}\en{Development of an interface for interaction with these automation tools using Flask and aiohttp};
        		\item \ru{Разработка скриптов и вспомогательных микросервисов на Python3}\en{Development of scripts and auxiliary microservices in Python3};
        		\item \ru{Интеграция компонент на C\texttt{++} и Python3 с использованием Cython}\en{Integrating C\texttt{++} and Python3 components using Cython};
                \item \ru{Портирование Legacy кода с Python2 на Python3}\en{Porting Legacy code from Python2 to Python3}.
        	\end{itemize}
        }

        \vspace{.5em}
    
        \entry{\interval{2022}{2022}}
        {\ru{Факультет Вычислительной Математики и Кибернетики Московского государственного университета имени М.В.Ломоносова}\en{CMC MSU} -- \ru{Разработчик}\en{Engineer}}
        {\ru{Разработка на Python}\en{Python development}: 
        	\begin{itemize}
        		\item \ru{Разработка автоматизированной изолированной системы проверки домашних заданий}\en{Development of an automated isolated system for homework checking};
                \item \ru{Разработка структуры системы}\en{Development of the system structure};
        		\item \ru{Имплиментация системы публикации результатов проверки на Python3 (Jinja2)}\en{Implementation of publishing system for test results in Python3 (Jinja2)}.
        	\end{itemize}
        }

        \vspace{.5em}
    
        % \entry {\interval{2023}{2024}}
        % {\ru{Репетитор по математике}\en{Math tutor}}
        % {\ru{Репетитор по математике 7--9 класс}\en{Math tutor for grades 7--9}}.
        % {\ru{Разработка на Python}\en{Python development}: 
        % 	\begin{itemize}
        % 		\item \ru{Разработка и поддержка инфраструктуры для торговли криптовалютами} \en{Development and maintenance of infrastructure for crypto trading}.
        % 		\item \ru{Создание}\en{Setting up} CI pipeline \ru{для}\en{for} CMake projects: building, testing, caching, automatic dependecies updating.
        % 		\item \ru{Имплементация протокола коммуникации между компонентами с использованием} \en{Implementing microservices communication protocol with} Aeron.
        % 		\item \ru{Разработка скриптов и вспомогательных микросервисов на} \en{Developing scripts and small microservices with} python3.
        % 	\end{itemize}
        % }

    % \section{\ru{Курсы}\en{Courses}}
    %     \entry {2021}
    %         {\ru{МФТИ}\en{MIPT} - \ru{Теория и практика многопоточной синхронизации}\en{Concurrency course} }
    %         {\ru{Язык C\texttt{++}}\en{C\texttt{++}}. \ru{Реализация библиотеки для работы с многопоточностью}\en{Implementation of simple concurrency library}: fibers, futures, thread pool.}
    
        % \vspace{.1cm}

    \vspace{.5em}

	\section{\ru{Образование}\en{Education}}
        \entry{\interval{2024}{2026}}
        {\ru{Московский государственный университет имени М.В.Ломоносова}\en{Lomonosov Moscow State University} \\
        \ru{Факультет вычислительной математики и кибернетики}\en{Faculty of Computational Mathematics and Cybernetics}\\
        \ru{Перспективные методы искусственного интеллекта в сетях передачи и обработки данных}\en{Advanced Methods of Artificial Intelligence in Data Transmission and Processing Networks}
        \ru{Кафедра автоматизации систем вычислительных комплексов}\en{Department of Computer Systems and Automation}
        }
        {\ru{Очная форма обучения, магистратура.}\en{Master's degree.}}

        \vspace{.5em}

        \entry{\interval{2020}{2024}}
        {\ru{Московский государственный университет имени М.В.Ломоносова}\en{Lomonosov Moscow State University} \\
        \ru{Факультет вычислительной математики и кибернетики}\en{Faculty of Computational Mathematics and Cybernetics}\\
        \ru{Кафедра автоматизации систем вычислительных комплексов}\en{Department of Computer Systems and Automation}
        }
        {\ru{Очная форма обучения, бакалавриат.}\en{Bachelor's degree.}}

    \vspace{.5em}
     
    \section{\ru{Проекты}\en{Projects}}
        \entry{\interval{2023}{2025}}
        {\ru{Реализация алгоритмов разбиения графов}\en{Implementation of graph partitioning algorithms}}
        {\ru{Реализация алгоритмов в рамках научной работы по теме распределения работ по процессорам}\en{Implementation of algorithms as part of scientific work on the distribution of workload across processors}.} 
    
        \vspace{.5em}
    
        \entry{2018}
        {\ru{Клиент-серверное приложение}\en{Client-Server Application}}
        {\ru{Сервер реализован на языке Python3 (Flask, SQLAlchemy);\\
        Клиент -- на языке Java}\en{The server is implemented in Python3 (Flask, SQLAlchemy)\\The client is implemented in Java}.} 
    
        % \vspace{.5em}

        % \entry {TODO}
        % {\ru{Преобразователь гитарных табов в пдф ???}\en{TODO}}
        % {\ru{TODO} 
        % \en{TODO}} 
    
        \vspace{.5em}

        \entry{2019}
        {\ru{Симулятор движения небесных тел}\en{Celestial bodies motion simulator}}
        {\ru{Решение задачи $n$ тел с помощью метода Рунге-Кутта 1-го порядка на Python3 с использованием NumPy}\en{Solving n-body problem with 1st order Runge-Kutta method in Python3 using NumPy}.}
    
        \vspace{.5em}

    \section{\ru{Достижения}\en{Achievements}}
        \begin{itemize}
            % \item \ru{Призёр регионального этапа ВСОШ по информатике}\en{Prize-winner of regional stage of the all-russian olympiad of school students in informatics}
            \item \ru{Победитель первой степени олимпиады ИТМО по информатике}\en{Winner of the open Olympiad of schoolchildren in informatic (ITMO)}
            \item \ru{Хакатон TulaHack - 4 место}\en{Hackathon TulaHack - 4th place}
        \end{itemize}

    % \vspace{.5em}
    % \section{\ru{Публикации}\en{TODO}}
    %     \begin{itemize}
    %         % \item \ru{Призёр регионального этапа ВСОШ по информатике}\en{Prize-winner of regional stage of the all-russian olympiad of school students in informatics}
    %         \item \ru{Победитель первой степени олимпиады ИТМО по информатике}\en{Winner of the open Olympiad of schoolchildren in informatic (ITMO)}
    %         \item \ru{Хакатон TulaHack - 4    место}\en{Hackathon TulaHack - 4th place}
    %     \end{itemize}

    
    \section{\ru{Навыки}\en{Skills}}
    	\begin{tabular}{ >{\bfseries}r | l }
    		\ru{Языки программирования}\en{Programming languages} & Python3, C/C\texttt{++}, NASM, SQL\\
    		% \ru{Технологии}\en{Technologies} & MongoDB, PostgreSQL, RabbitMQ, Kafka  \\
    		\ru{Фреймворки и библиотеки}\en{Frameworks and libraries} & Flask, asyncio, aiohttp, NumPy, Jinja2\\
    		\ru{Инструменты}\en{Tools} & Git, CMake, Docker\\
    		\ru{Языки}\en{Languages} & Russian(Native), English(B2)\\
            \ru{Прочее}\en{Other skills} & LaTeX\\
    	\end{tabular} 
        
    \vspace{\fill}
    \begin{center}
        \large
        \href{https://github.com/Marilius/CV}{\ru{Актуальная версия этого резюме}\en{Up to date version of this CV}}
    \end{center}
\end{document}